\documentclass{article}
\usepackage[utf8]{inputenc}
\usepackage[spanish]{babel}
\usepackage{graphicx}
\usepackage{hyperref}
\usepackage{geometry}
\usepackage{parskip}
\usepackage{lipsum} % Para texto de ejemplo

\title{Metaheurísticas: Práctica 1}
\author{Ángel Sánchez Guerrero}
\date{Marzo 2025}

\begin{document}
\thispagestyle{empty}

% Logo UGR arriba
\begin{center}
    \includegraphics[width=12cm]{logo_ugr.jpg}
\end{center}

\vspace{2cm}

% Título de asignatura
\begin{center}
    {\Huge \textbf{Metaheurísticas}} \\[3em]
    {\LARGE \textbf{Práctica Nº 1}} \\[1em]
    {\Large Problema de Mínima Dispersion Diferencial (MDDP)} \\[3em]

    {\large Curso Académico: \textbf{2024/2025}} \\[3em]

    {\large \textbf{Horario de prácticas:} Jueves 15:30 – 17:30} \\
    {\large \textbf{Email:} angel.sanchez01@correo.ugr.es} \\
    {\large \textbf{Nombre:} Ángel Sánchez Guerrero} \\
    {\large \textbf{DNI:} 12345678A} \\
\end{center}

\vfill

% Logo ETSIIT abajo
\begin{center}
    \includegraphics[width=4cm]{etsiit_logo.png}
\end{center}

\newpage
\tableofcontents
\newpage

\section{Descripción y formulación del problema}

El Problema de la Mínima Dispersión Diferencial (Minimum Differential Dispersion Problem, MDDP) es un problema de optimización combinatoria NP-completo. Su formulación es aparentemente sencilla, pero su resolución resulta computacionalmente compleja incluso para instancias de tamaño moderado, superando la hora de cómputo para casos de tamaño 50.

El problema consiste en seleccionar un subconjunto \(M\) de \(m\) elementos de un conjunto inicial \(S\) con \(n\) elementos (donde \(n > m\)), de forma que se minimice la dispersión entre los elementos escogidos. Para cada par de elementos, se conoce la distancia entre ellos, representada en una matriz \(D = (d_{ij})\) de dimensión \(n \times n\).

En el caso específico del MDDP, la medida de dispersión se calcula de la siguiente manera:
\begin{enumerate}
    \item Para cada elemento \(v\) seleccionado, se calcula su valor \(\Delta(v)\) como la suma de las distancias de este elemento al resto de elementos seleccionados.
    \item La dispersión de una solución, denotada como \(diff(S)\), se define como la diferencia entre los valores extremos de \(\Delta\):
    \[diff(S) = \max\{\Delta(v) : v \in S\} - \min\{\Delta(v) : v \in S\}\]
    \item El objetivo es minimizar esta medida de dispersión:
    \[\min diff(S)\]
\end{enumerate}

Formalmente, el problema se puede formular mediante la siguiente expresión:
\[\text{Minimizar } \max_{x_i \in M} \left( \sum_{j \in M} d_{ij} \right) - \min_{x_i \in M} \left( \sum_{j \in M} d_{ij} \right) \text{ con } M \subset S, |M| = m\]

Este problema tiene múltiples aplicaciones prácticas, como:
\begin{itemize}
    \item Ubicación óptima de instalaciones públicas (como farmacias, hospitales o estaciones de servicio)
    \item Selección de grupos homogéneos en entornos corporativos o educativos
    \item Identificación de redes densas en análisis de grafos
    \item Reparto equitativo de recursos
    \item Problemas de flujo en redes de transporte o comunicación
\end{itemize}

Debido a su complejidad computacional, resulta necesario emplear métodos aproximados para su resolución, como algoritmos voraces (greedy), búsqueda local o metaheurísticas más avanzadas, que permitan obtener soluciones de calidad en un tiempo razonable.

\end{document}
